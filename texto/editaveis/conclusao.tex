\chapter[Conclusão]{Conclusão}

A evolução dos equipamentos relacionados à mixagem de \textit{DJs} anda de mãos dadas com o próprio desenvolvimento da eletrônica como um campo maior, sendo as inovações feitas por pessoas que se interessavam tanto pela eletrônica quanto por música, e a partir de suas vontades e anseios, decidiam desenvolver novas soluções com o que a eletrônica oferecia.

Assim, esse trabalho visa a criação de um \textit{mixer} que lesse dois arquivos \textit{wavs} localizados dentro de uma \textit{Raspberry Pi}, de forma que um potenciômetro fosse o responsável pelo controle de duas frequências de corte, cada uma para um canal de áudio. Além disso, o projeto integrou dois efeitos: \textit{delay} e \textit{reverb} ao processamento desses sinais. E por fim, o sinal resultante visava ser reproduzido em um sistema de áudio através da interface de áudio via cabo 3,5 mm já integrado à placa.

Os resultados obtidos contribuíram para a validação de um sistema unificado de filtragem de sinais de áudio a partir de apenas um controle. Os sinais obtidos em relação à filtragem demonstraram um bom funcionamento visto que, para cada frequência de corte, espera-se que determinados elementos na composição de uma música sejam atenuados, o que torna a transição entre uma música e outra mais suave e natural.

Além disso, os resultados obtidos em relação aos efeitos permitiram com que, através de um código simples, a implementação gerasse resultados satisfatórios, dando ao usuário a opção de escolher a presença do efeito ao sinal final. 

Esse trabalho tem a potência de contribuir para uma área de prototipagem de equipamentos eletrônicos que tem ganhado espaço no mercado relacionados a pequenos \textit{makers}. Esse é um espaço que tem crescido e coexistido ao lado de grandes mercados existentes.

No entanto, essa pesquisa e desenvolvimento se limitou a utilizar sinais advindos de arquivos \textit{wav}. Dessa forma, por não utilizar entradas de equipamentos como \textit{CDJs} ou toca-discos, sua utilização é menos ampla do que poderia ser. Além disso, as implementações da filtragem e efeitos utilizaram certas configurações e lógicas, de forma que esses resultados devem ser restringidos ao serem analisados a outras configurações ou plataformas.

Trabalhos futuros podem utilizar a entrada de sinais advindos de equipamentos e explorar a otimização do código ao utilizar \textit{threads} e sincronizações de chamadas às funções. Além disso, inúmeras formas de mixagem ou efeitos podem ser desenvolvidos e encapsulados em um dispositivo portátil como uma \textit{Raspberry Pi}.

Em suma, esse trabalho visa alimentar a inovação em um ramo da eletrônica voltada à música cuja presença tem aumentado, modificando formas usuais através das quais se tem estabelecido do que é uma mixagem e do que ela pode ser, sendo uma ferramenta para potencializar a criatividade do artista diante da plateia. 