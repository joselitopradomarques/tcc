\chapter[Conclusão]{Conclusão}

Este projeto visa o desenvolvimento de um dispositivo para \textit{DJs} que permita a leitura de sinais provenientes de reprodutores de música e a realização de mixagens automáticas. O dispositivo modificará a frequência de corte de dois filtros passa-altas para realizar a transição entre músicas de forma suave e direta. Além disso, serão incorporados dois efeitos: \textit{reverb} e \textit{delay}, com parâmetros ajustáveis, como dB após 1s e ms de atraso.

O equipamento utilizará uma \textit{Raspberry Pi} e um \textit{Arduino Uno} para leitura, processamento de sinais e conversões necessárias. Toda a lógica de processamento será implementada em linguagem C para garantir alta velocidade e minimizar a latência.

O dispositivo será montado em uma caixa com uma interface de usuário composta por botões \textit{sliders} horizontais e um botão de duas posições.

Para assegurar a usabilidade efetiva do equipamento, serão realizados testes detalhados com usuários e coletados \textit{feedbacks} para aprimorar o desempenho e a funcionalidade dos filtros e efeitos.
