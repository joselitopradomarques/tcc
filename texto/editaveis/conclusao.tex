\chapter[Conclusão]{Conclusão}


Esse projeto almeja o desenvolvimento de um dispositivo para \textit{DJs} que consiga ler sinais advindo de reprodutores de música e realizar a mixagem, modificando automaticamente a frequência de corte de dois filtros passa-altas, de forma que a transição entre as duas músicas seja realizada de forma direta. Além disso, pretende-se a incorporação de dois efeitos: \textit{reverb} e \textit{delay}, com seus respectivos parâmetros: dB após 1s e ms de atraso, configuráveis. 

Esse equipamento utilizará uma \textit{Raspberry Pi} e um \textit{Arduino Uno} para a leitura e processamento de sinais, bem como as conversões necessárias. Toda lógica será implementada utilizando a linguagem C, visando velocidade no processamento de forma que a latência seja minimizada.

O dispositivo será encapsulado em uma caixa com botões que realizarão a interface com o usuário através de botões \textit{sliders} horizontais e um botão de duas posições.

Para efetiva usabilidade do equipamento, detalhamentos acerca dos filtros utilizados e testes com usuários com seus respectivos \textit{feedbacks} serão realizados.