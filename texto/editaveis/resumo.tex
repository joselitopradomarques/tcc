\begin{resumo}
    A história que permeia \textit{mixer}, equipamento essencial para a atividade realizada por um \textit{dj}, consagra um único modelo de mixagem baseado no controle de ganho de três bandas. Assim, o desenvolvimento de um \textit{mixer} que controla dois canais utilizando apenas um controle é realizado. A transição entre duas músicas é otimizada pela utilização de dois filtros passa-altas, que são deslocados conforme um botão central é utilizado. O equipamento é capaz de realizar a leitura de sinais analógicos advindo de reprodutores de música consagrados pela indústria como \textit{CDJs}. Incremento de opções de efeitos como \textit{reverb} e \textit{delay} são feitos, bem como o controle de seus parâmetros de implementação são controláveis através de um botão \textit{slider} horizontal. Esse sistema é implementado utilizando uma \textit{Raspberry Pi} para leitura, escrita e processamento de sinais bem como a lógica implementada em C visando otimização da latência. Prova de conceito foram realizadas através do \textit{PureData} tanto para a filtragem de sinais e funcionamento dos filtros passa-altas em função do posicionamento do botão, que dita a posição das frequências de corte dos filtros, além do controle dos parâmetros dos efeitos como ganho dos sinais amostrados após 1s para o efeito \textit{reverb} e atraso das amostras em ms para o \textit{delay}.

 \vspace{\onelineskip}
    
 \noindent
 \textbf{Palavras-chave}: mixer. dj. raspberrypi.
\end{resumo}
