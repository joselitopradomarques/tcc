\begin{resumo}

    Um \textit{disk-jockey} (DJ) é um artista que cria uma nova experiência sonora a partir de músicas pré-existentes. Devido às suas habilidades de analisar os componentes presentes na música e utilizar equipamentos específicos, esse artista é capaz de transformar a experiência de simplesmente ouvir uma sequência de músicas.
    A principal ferramenta do DJ para esse fim é o \textit{mixer}, que é capaz de transformar cada música a partir de funcionalidades como filtragem e efeitos.

    %Um \textit{DJ} cria uma nova experiência sonora a partir de músicas que estão disponíveis a todos, mas, devido às suas habilidades de analisar os componentes presentes na música e utilizar equipamentos específicos, esse artista é capaz de transformar a experiência de simplesmente ouvir uma sequência de música, principalmente utilizando um \textit{mixer}, um equipamento capaz de transformar cada música, a partir de funcionalidades como filtragem e efeitos. O responsável por fazer esse intermédio é o \textit{mixer}, um equipamento que dá o controle desses elementos ao \textit{DJ}. Porém, percebeu-se um padrão na forma como \textit{DJs} realizam mixagens, e, ao invés de utilizar 3 botões para cada canal, decidiu-se criar um sistema capaz de unificar esse controle, para simplificar a forma como esse processo é realizado.

    Uma das formas de fazer a transição entre músicas de dois canais do \textit{mixer} é utilizando um filtro passa-altas e o controle de volume. Isso requer do DJ o acionamento de até 3 botões para cada canal, o que pode ser difícil de executar manualmente. Neste trabalho, decidiu-se criar um sistema capaz de unificar esse controle, para simplificar a forma como esse processo é realizado.
    Com isso, a criação de um dispositivo capaz de realizar transições a partir de um único botão que controla filtros \textit{passa-altas} para cada canal utilizado foi implementado utilizando uma \textit{Raspberry Pi} e simples potenciômetros. Além disso, efeitos como \textit{delay} e \textit{reverb} foram implementados para dar mais liberdade de criação ao \textit{DJ}.
    
    % Para isso, foi necessário o desenvolvimento do zero de cada etapa necessária, desde a obtenção dos sinais, passando pelas etapas de filtragem e aplicação de efeitos, até a conversão desses sinais para a sua reprodução em um sistema de áudio.
    
    Com isso, foi possível a obtenção de um protótipo de um \textit{mixer} que realize a transição entre duas faixas utilizando um botão de controle que modifica todo um sistema de filtros que controla de forma automática os elementos que compõem as faixas.
    Essa solução evidenciou, através de testes feitos nos sinais obtidos a partir da variação de parâmetros, de forma a varrer as novas possibilidades criadas a partir dessa nova configuração, a possibilidade de criação de dispositivos voltados a \textit{DJs} criados a partir de componentes simples como \textit{Raspberry Pi} e conceitos primordiais da teoria de processamento de sinais, estruturas que vão na contramão do que se é encontrado hoje na indústria.
    
    % Assim, o dispositivo criado evidencia que, a partir de dispositivos comuns e conceitos amplamente difundidos, é possível a criação de novas ferramentas que venham a expandir a atuação de um \textit{DJ} explorando sua criatividade em campos não explorados.

 \vspace{\onelineskip}
    
 \noindent
 \textbf{Palavras-chave}: processamento de sinal; áudio; mixer.
\end{resumo}
