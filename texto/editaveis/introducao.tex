\chapter[Introdução]{Introdução}

A música eletrônica e a cultura \textit{DJ} possuem um papel fundamental no cenário musical contemporâneo, ampliando a forma como a música pode ser apreciada. O desenvolvimento de tecnologias específicas para esse público caminha lado a lado com a evolução da eletrônica, permitindo que a habilidade de selecionar, alterar e misturar músicas ofereça novas experiências ao público. Nesse contexto, o \textit{mixer} se destaca como o dispositivo central na mixagem de sons, processando e combinando múltiplos canais de áudio para criar uma nova composição sonora. A crescente demanda por equipamentos de áudio que permitam maior controle criativo tem impulsionado o desenvolvimento de novas tecnologias de mixagem, porém, as novas soluções surgidas têm como norte a integração a telas, expansão de quantidade de canais e acesso à nuvem, de forma que a lógica basal de mixagem se encontra enrijescida.

Dessa forma, esse projeto visa a exploração da lógica de mixagem, basal para a expansão da compreensão do processo criativo do \textit{DJ}. Ao buscar novas formas de personalização e controle, o protótipo desenvolvido propõe um \textit{mixer} mais flexível, que se adapta às necessidades específicas dos profissionais, permitindo uma maior interação com os elementos sonoros e oferecendo novas possibilidades para a criação musical.

\section{Justificativa}
O mercado de equipamentos para \textit{DJs} é amplamente dominado por grandes empresas, mas existe uma crescente demanda por dispositivos que fogem à norma estabelecida. Nesse contexto, surgem pequenas empresas e \textit{boutiques} especializadas que desenvolvem soluções inovadoras, atendendo a \textit{DJs} que buscam mais controle criativo e flexibilidade em suas performances. Percebe-se que há uma necessidade crescente de equipamentos que ofereçam mais versatilidade e personalização, permitindo aos \textit{DJs} explorar suas habilidades de forma mais livre, mantendo, no entanto, a praticidade e aplicabilidade. A solução proposta busca responder a essa demanda por novos recursos, oferecendo uma abordagem simples e funcional, que se adapta facilmente ao uso cotidiano de um \textit{DJ}.

Através da análise do processo de transição entre faixas, observou-se que os \textit{mixers} tradicionais funcionam com a manipulação independente dos controles de frequências para cada canal. A transição típica envolve a diminuição dos graves de uma faixa, enquanto aumenta-se os elementos agudos da próxima faixa, de modo a suavizar a mudança diferentes elementos para que não haja uma variação brusca nos elementos. Ao perceber esse padrão de manipulação, surgiu a ideia de unificar os controles, permitindo um controle mais intuitivo e direto, sem perder a capacidade de personalização.

Com essa unificação, a solução proposta modifica o conceito tradicional de controle de frequências. Em vez de ajustar bandas de frequência separadamente, a ideia é utilizar um controle único de corte de frequências, que atuaria de forma simultânea nos dois canais, permitindo uma transição mais fluida e simplificada. Essa abordagem não só facilita o trabalho do \textit{DJ}, mas também oferece uma nova maneira de explorar as possibilidades sonoras durante a performance. Essa inovação visa, portanto, ampliar as opções criativas disponíveis de forma que possa ser facilmente aplicável a um sistema atual. 

\section{Objetivos}

Nessa seção, o objetivo geral e os objetivos específicos almejados por esse projeto são citados.

\subsection{Objetivo Geral}
Esse projeto visa a criação de um \textit{mixer} implementado em um sistema embarcado que unifique o controle da mixagem feita por \textit{DJ}s em contrapartida ao modelo tradicional de mixagem baseado em equalização de três bandas.

\subsection{Objetivos Específicos}

Quanto aos objetivos específicos, elencou-se pontos ao longo do desenvolvimento desse sistemas que servirão como pontos de referência para garantir que o funcionamento do todo. 

\begin{enumerate}
    \item \label{item:deslocamento} Realizar de forma automatizada o controle da frequência de corte de dois canais a partir de um controle
    \item \label{item:efeitos} Implementar dois tipos de efeitos
    \item \label{item:conversao} Converter o sinal obtido após o processamento para analógico para ser reproduzido por um sistema de som
\end{enumerate}

O Objetivo Específico \ref{item:deslocamento} envolve a obtenção dos sinais de controle, conversões de analógico para digitais, a atuação correta nos filtros passa-altas de ambos canais.

Em relação aos efeitos, o Objetivo Específico \ref{item:efeitos} visa embarcar desde a concepção matemática, implementação do código para cada efeito e sua implementação para que o efeito seja aplicado ao sinal.

Uma vez que a conclusão dos Objetivos acima garante um sinal processado no domínio digital, é necessário garantir que o sinal seja adequadamente convertido para analógico, a fim de ser reproduzido em equipamentos externos. Portanto, o Objetivo Específico \ref{item:conversao} agrega a conversão do sinal de saída para que o sistema de som conectado ao sistema consiga reproduzir o canal de saída.

\section{Estrutura do Documento}

O presente trabalho está organizado em capítulos que abordam os diferentes aspectos da história, desenvolvimento e implementação do protótipo de \textit{mixer} para \textit{DJs} com controle unificado de frequência. A seguir, é apresentada a estrutura do documento.

% O primeiro capítulo, \textbf{Introdução}, faz um delineamento do contexto no qual surgiu a demanda por novas abordagens. Nele, há uma descrição sobre o mercado de equipamentos focados em \textit{DJs} e uma breve explicação sobre como a mixagem é realizada e como esse processo pode ser automatizado. Em seguida, são apresentados os objetivos geral e específicos que guiarão as etapas subsequentes, seguidos de uma descrição sobre a estrutura do texto, detalhando o que cada capítulo aborda para garantir um documento coeso.
No segundo capítulo, \textbf{Fundamentação Teórica e Estado da Arte}, são explorados os conceitos físicos e matemáticos fundamentais, como sinais e sistemas nos domínios contínuo e discreto, % transformações matemáticas e conceitos essenciais para a transição entre os mundos analógico e digital, fundamentais para o tratamento dos sinais provenientes do mundo analógico, que precisam ser convertidos novamente para esse formato. Em seguida, são apresentadas explicações teóricas sobre funcionalidades como filtragem. O capítulo continua explorando
bem como o som, tanto do ponto de vista físico quanto da percepção humana dessa grandeza. Também é realizada uma pesquisa histórica sobre a evolução da mixagem e dos dispositivos utilizados, oferecendo uma compreensão sobre a magnitude e a atuação de um \textit{DJ}, culminando nas formas contemporâneas de mixagem.
% Ainda no segundo capítulo, é detalhada a maneira como o som é trabalhado, destacando como os sinais podem ser compreendidos a partir de duas bandas e como esse parâmetro se relaciona com os elementos que compõem uma música. Além disso, uma pesquisa técnica sobre as conexões entre dispositivos é apresentada, a fim de consolidar a forma como um dispositivo pode interagir com o mundo real.
% Em seguida, 

No terceiro capítulo, \textbf{Metodologia}, a proposta do projeto é descrita a partir de requisitos funcionais e não-funcionais, %, conceitos emprestados da engenharia de \textit{software}. O capítulo se divide 
em duas grandes subseções: prova de conceito e proposta de implementação.
Na prova de conceito, a proposta foi testada utilizando um \textit{software} chamado \textit{PureData}, no qual a lógica de filtragem e efeitos, bem como suas interações, foram implementadas. % Diagramas e descrições detalhadas de cada etapa do processamento são apresentados.
Na proposta de implementação, a solução final é explorada com o detalhamento dos componentes eletrônicos e da solução \textit{software} utilizados.

Para validar tanto a prova de conceito quanto a proposta de implementação, o capítulo \textbf{Resultados} analisa os sinais obtidos a partir da variação dos parâmetros de entrada, de modo a validar o comportamento do sistema para além da simples escuta do sinal final gerado. %Para isso, gráficos no domínio do tempo e da frequência foram utilizados. Com esses resultados, foi possível validar o bom funcionamento da configuração dos filtros utilizados, bem como dos efeitos implementados.

Com o projeto implementado, o capítulo \textbf{Conclusão} realiza uma visão geral de todo o trabalho, destacando a integração de todas as etapas que compuseram o projeto. %Nessa seção, a implementação de um sistema que propõe a alteração e unificação do sistema de mixagem tradicionalmente utilizado pela indústria foi validada por meio da análise dos resultados obtidos.