\chapter[Introdução]{Introdução}

Sons são formados por elementos que estão presentes em faixas de frequências. Muitas vezes, deseja-se o controle desses elementos. Assim, a manipulação dessas bandas é realizada através de filtros com ganhos e atenuações. O equipamento responsável por esse manuseio é o \textit{mixer}.
Ao se analisar a história desse equipamento, percebe-se que a forma de mixagem parte do controle de bandas de frequência. Inicialmente, cortes bruscos entre canais eram realizados e reproduções eram feitas utilizando toca-discos com um microfone captando esse sinal.

Ao longo da segunda metade do século XX, o progresso da eletrônica permitiu a sofisticação de cada processo envolvido no ato de mixagem. Inicialmente, utilizou-se vinis e hoje se pode utilizar um banco de músicas hospedado na nuvem enquanto se utiliza óculos de realidade aumentada para realizar a mixagem.

O processo de mixagem evoluiu ao longo do tempo, porém a ideia de utilizar recortes de músicas para criar uma ambientação única permanece e permeia toda a história, e norteia o futuro do trabalho realizado por um \textit{DJ}.

\section{Justificativa}

Ao se analisar o desenvolvimento dos equipamentos utilizados para mixar músicas, percebe-se uma constante quanto aos parâmetros utilizados para realizar transições entre músicas, que reside na alteração do ganho de bandas de frequências, usualmente três: baixas, médias e altas frequências.

Além disso, o uso combinado de filtragem e \textit{crossfader}r é amplamente empregado na prática de mixagem. Essa combinação pode simplificar o trabalho do \textit{DJ}, proporcionando uma transição mais suave e eficiente entre as faixas. 

A proposta de utilizar um botão central para controlar dois canais oferece uma vantagem adicional ao permitir um controle mais intuitivo e centralizado dos efeitos aplicados. Isso pode facilitar a aplicação e ajuste de efeitos de forma mais direta e rápida, aumentando a flexibilidade e a criatividade durante a mixagem.

\section{Objetivos}

Nessa seção, o objetivo geral e os objetivos específicos almejados por esse projeto são citados.

\subsection{Objetivo Geral}
Esse projeto visa a criação de um \textit{mixer} implementado em um sistema embarcado que facilite a mixagem de \textit{DJ}s tendo em vista os equipamentos disponíveis hoje no mercado.

\subsection{Objetivos Específicos}
\begin{enumerate}
    \item Obter leituras de sinais analógicos de equipamentos como \textit{CDJs}
    \item Realizar de forma automatizada o deslocamento de frequências de corte dos filtros passa-altas
    \item Implementar dois tipos de efeitos
    \item Converter o sinal obtido após o processamento para analógico a fim de ser reproduzido por um sistema de som
    \item Desenvolvimento de um sistema com uma interface amigável para um \textit{DJ}
\end{enumerate}

\section{Estrutura do Documento}

Esse documento parte do conceito de sinal e sistema para que o processo de amostragem e reconstrução de um sinal seja realizado. Posteriormente, uma conceituação acerca de som e música é realizada percorrendo o conceito físico e uma definição expandida utilizando o pensamento de John Cage.

Além disso, uma retrospectiva do desenvolvimento de \textit{mixer}s utilizados por \textit{DJ}s desde sua criação até o momento presente é realizada, bem como um detalhamento acerca da interface utilizada, além dos conectores usualmente utilizados. Funcionalidades usuais de um \textit{mixer} são levantadas para que se delimitem os processos essenciais de um \textit{mixer}.

Em seguida, a metodologia do projeto é detalhada com um levantamento de requisitos do projeto. Um fluxograma geral e seus subfluxogramas são detalhados para que cada função fique clara quanto ao seu funcionamento.

Uma seção correspondente à prova de conceito se encontra presente, na qual uma simulação do sistema foi realizada contando com suas almejadas funções de filtragem e efeitos.

Para clarificar a proposta de implementação desse projeto, explanações acerca da interface de usuário e do processamento do sinal são realizadas.

Ao fim, os resultados obtidos da simulação do sistema foram explanados e a conclusão do objetivo final esperado para esse projeto foi realizada.
