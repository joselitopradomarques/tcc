\begin{resumo}[Abstract]
  \begin{otherlanguage*}{english}
  \textit{A disk-jockey (DJ) is an artist who creates a new sound experience from pre-existing music. Due to their ability to analyze the components present in music and use specific equipment, this artist is able to transform the experience of simply listening to a sequence of songs.
The DJ's main tool for this purpose is the mixer, which is capable of transforming each song using features such as filtering and effects.}

\textit{One way to transition between songs on two mixer channels is by using a high-pass filter and the volume control. This requires the DJ to press up to 3 buttons for each channel, which can be difficult to do manually. In this work, we decided to create a system capable of unifying this control, to simplify the way this process is carried out. With this, the creation of a device capable of performing transitions from a single button that controls} high-pass \textit{filters for each channel used was implemented using a Raspberry Pi and simple potentiometers. In addition, effects such as delay and reverb were implemented to give the DJ more creative freedom.}

\textit{With this, it was possible to obtain a prototype of a mixer that performs the transition between two tracks using a control button that modifies an entire filter system that automatically controls the elements that make up the tracks. This solution demonstrated, through tests carried out on signals obtained from parameter variations, in order to explore the new possibilities created from this new configuration, the possibility of creating devices aimed at DJs created from simple components such as Raspberry Pi and fundamental concepts of signal processing theory, structures that go against what is found in the industry today.}
    % \textit{A \textit{DJ} creates a new sound experience from songs that are available to everyone, but due to their skills in analyzing the components present in the music and using specific equipment, this artist is able to transform the experience of simply listening to a sequence of music, mainly using a \textit{mixer}, an equipment capable of transforming each song through functionalities such as filtering and effects. The one responsible for mediating this process is the \textit{mixer}, a piece of equipment that gives control of these elements to the \textit{DJ}. However, a pattern in how \textit{DJs} perform their mixes was noticed, and instead of using three buttons for each channel, it was decided to create a system capable of unifying this control to simplify the process.\\ Thus, the creation of a device capable of performing transitions through a single button that controls \textit{high-pass} filters for each used channel was implemented using a \textit{Raspberry Pi} and simple potentiometers. Moreover, effects such as \textit{delay} and \textit{reverb} were implemented to give the \textit{DJ} more creative freedom.\\ For this, it was necessary to develop from scratch each required step, from obtaining the signals, passing through filtering stages and applying effects, to converting these signals for their reproduction in an audio system.\\ As a result, a prototype of a \textit{mixer} was obtained that performs the transition between two tracks using a control button that modifies an entire filter system that automatically controls the elements that make up the tracks.\\ This solution evidenced, through tests conducted on signals obtained from varying parameters to sweep the new possibilities created by this new configuration, the possibility of creating devices aimed at \textit{DJs}, created from simple components like the \textit{Raspberry Pi} and fundamental concepts of signal processing theory, structures that go against what is typically found in today's industry. \\ Thus, the created device demonstrates that, starting from common devices and widely diffused concepts, it is possible to create new tools that expand the role of a \textit{DJ}, exploring their creativity in uncharted fields.} \\ 
    \\
    \textbf{\textit{Keywords}}: \textit{signal processing, audio, mixer}.
  \end{otherlanguage*}
 \end{resumo}
 