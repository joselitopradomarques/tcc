\begin{resumo}[Abstract]
  \begin{otherlanguage*}{english}
  \textit{A disk-jockey (DJ) is an artist who creates a new sound experience from pre-existing music. Due to their ability to analyze the components present in music and use specific equipment, this artist is able to transform the experience of simply listening to a sequence of songs.
The DJ's main tool for this purpose is the mixer, which is capable of transforming each song using features such as filtering and effects.One way to transition between songs on two mixer channels is by using a high-pass filter and the volume control. This requires the DJ to press up to 3 buttons for each channel, which can be difficult to do manually. In this work, we decided to create a system capable of unifying this control, to simplify the way this process is carried out. With this, the creation of a device capable of performing transitions from a single button that controls high-pass filters for each channel used was implemented using a Raspberry Pi and simple potentiometers. In addition, effects such as delay and reverb were implemented to give the DJ more creative freedom. With this, it was possible to obtain a prototype of a mixer that performs the transition between two tracks using a control button that modifies an entire filter system that automatically controls the elements that make up the tracks. This solution demonstrated, through tests carried out on signals obtained from parameter variations, in order to explore the new possibilities created from this new configuration, the possibility of creating devices aimed at DJs created from simple components such as Raspberry Pi and fundamental concepts of signal processing theory, structures that go against what is found in the industry today.}

\textbf{\textit{Keywords}}: \textit{signal processing, audio, mixer}.
  \end{otherlanguage*}
 \end{resumo}
 