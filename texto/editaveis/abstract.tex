\begin{resumo}[Abstract]
 \begin{otherlanguage*}{english}
  \textit{An audio signal has unique characteristics of elements that are present in specific frequency ranges. And the sum of these elements makes up a piece of music. Thus, it is necessary to have control over the frequency bands that make up a signal. The \textit{mixer} is the equipment responsible for controlling the gains and attenuations of these bands. Its applications began in telephony, went through musical production desks, radio stations, and eventually reached the DJ's sound desk. The \textit{mixer} focused on a DJ's activity was consolidated from a single model based on the control of three-band gain. Thus, the development of a \textit{mixer} that controls two channels using only one control is carried out. The transition between two songs is optimized by using two high-pass filters, which are shifted as a central button is used. The equipment is capable of reading analog signals from music players standardized by the industry, such as \textit{CDJs}. Incremental options for effects such as \textit{reverb} and \textit{delay} are included, as well as the control of their implementation parameters is adjustable through a horizontal \textit{slider} button. This system is implemented using a \textit{Raspberry Pi} for reading, writing, and processing signals, as well as the logic implemented in C aiming at optimizing latency. Proofs of concept were carried out using \textit{PureData} both for signal filtering and the functioning of the high-pass filters based on the button's positioning, which dictates the cutoff frequencies of the filters, in addition to controlling the parameters of the effects such as signal gain sampled after 1s for the \textit{reverb} effect and sample delay in ms for the \textit{delay}.}
    
   \textbf{Key-words}: \textit{signal processing, audio, mixer.}
 \end{otherlanguage*}
\end{resumo}
