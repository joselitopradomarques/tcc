\begin{resumo}[Abstract]
 \begin{otherlanguage*}{english}
   The history surrounding mixers, essential equipment for a DJ's activity, establishes a single mixing model based on three-band gain control. Thus, the development of a mixer that controls two channels using only one control is carried out. The transition between two songs is optimized by using two high-pass filters, which are shifted as a central button is used. The equipment is capable of reading analog signals from music players renowned in the industry, such as CDJs. Options for effects like reverb and delay are added, and the control of their implementation parameters is manageable through a horizontal slider button. This system is implemented using a Raspberry Pi for reading, writing, and signal processing, as well as logic implemented in C to optimize latency. Proof of concept was carried out through PureData, both for signal filtering and the operation of the high-pass filters depending on the button position, which dictates the filters' cutoff frequencies, and for controlling effect parameters, such as the gain of sampled signals after 1 second for the reverb effect and the delay of samples in milliseconds for the delay effect.


   \textbf{Key-words}: mixer. dj. raspberrypi.
 \end{otherlanguage*}
\end{resumo}
