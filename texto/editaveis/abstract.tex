\begin{resumo}[Abstract]
 \begin{otherlanguage*}{english}
  \textit{
    An audio signal has unique characteristics due to elements that are present in specific frequency ranges. The sum of these elements constitutes a piece of music. Therefore, it is necessary to have control over the frequency bands that make up a signal. The mixer is the equipment responsible for controlling the gains and attenuations of these bands. Its applications began in telephony, evolved to music production consoles, radio stations, and eventually reached a DJ’s mixing desk. The mixer focused on DJ activities became established with a single model based on three-band gain control. Thus, the development of a mixer that controls two channels using only one control is carried out. The transition between two tracks is optimized by using two high-pass filters that are adjusted as a central knob is used. The equipment is capable of reading analog signals from music players standardized by the industry, such as CDJs. The system includes additional effects options like reverb and delay, with their implementation parameters being controllable through a horizontal slider knob. This system is implemented using a Raspberry Pi for reading, writing, and processing signals, as well as the logic implemented in C to optimize latency. Proof of concept was conducted using PureData, both for signal filtering and for the operation of the high-pass filters based on the knob position, which dictates the filters' cutoff frequencies, as well as for controlling effect parameters such as gain of the sampled signals after 1 second for the reverb effect and sample delay in milliseconds for the delay effect.
    }
    
   \textbf{Key-words}: \textit{signal processing, audio, mixer.}
 \end{otherlanguage*}
\end{resumo}
