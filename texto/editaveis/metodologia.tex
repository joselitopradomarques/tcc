\chapter[Metodologia]{Metodologia}


\section{Proposta Geral}
Ao analisar a história de mixers e equipamentos e o mercado atual, chega-se a conclusão de que a evolução dos equipamentos foi e é realizada partindo de um modelo que se baseia na mixagem feita em vinil. Ao longo do tempo, acrescentou-se novas funcionalidades e formas de interações, porém, na maioria das vezes, mantendo o jogger para atraso e avanço da música, \textit{fadder} para volume e knobs para frequências.
\par
Além disso, conforme a evolução da microeletrônica, equipamentos começaram a migrar aos poucos para a eletrônica digital, culminando na utilização de um DSP para o processamento do sinal de áudio para comportar formatos de música com baixa compressão como WAV e FLAC. 
\par
Assim, o que se vê hoje são dois extremos: equipamentos extremamente caros mas que não necessitam da presença de um computador e equipamentos dependentes, que são mais baratos e acabam possuindo melhor qualidade nos processamentos, que dificultam a mobilidade.
\par
Além disso, em experiência adquiridas com \textit{DJs}, observa-se que uma das maiores dificuldades que iniciantes possuem é realizar um ajuste fino entre as bandas de frequências quando lidam com \textit{mixers} clássicos de ajustes de três bandas. E é essa habilidade que permite o artista de destacar elementos de músicas e criar uma nova atmosfera. Apesar disso, quando o estilo de gênero de um \textit{DJ} está consolidad em uma área, a mixagem se torna mais mecânica, de forma que os elementos entre as música selecionadas se tornam semelhantes, o que permite maior flexibilidade na alteração de parâmetros de mixagem. Ou seja, há menos margem para que uma colagem sonora feita se torne desagradável aos ouvidos.
\par
Dessa forma, o intuito é conceber um \textit{mixer} que conte com um controle central, que visa controlar duas músicas, ou dois \textit{decks}. Em contrapartida à estrutura clássica de mixagem, que conta com controles de ganho de três bandas de frequência para cada canal, a proposta visa a utilização de um filtro passa-altas para cada canal. 
\par
O dispostivo conta com dois canais de entrada, ou seja, duas músicas de entrada. Conforme um filtro está passando toda a banda de um canal, o outro filtro automaticamente está rejeitando o outro canal. E conforme o controle central avança em direção ao canal dois, a frequência de corte do primeiro canal aumenta, diminuindo a presença da música um, enquanto a frequência de corte do canal dois diminui, aumentando a presença de elementos da música dois. 
\par
Assim, um controle central de frequência comanda ambas frequências de corte. Quando esse controle está em um extremo, apenas um canal é transmitido para a saída. Já no outro extremo, apenas o outro canal é transmitido. E conforme o controle avança de um extremo a outro, uma combinação entre as duas músicas é transmitida à saída.
\par
Alguns equipamentos atuais contam com ferramentas de mixagem automática, na qualuma lista de músicas é selecionada. O próprio sistema realiza a sincronia de tons e a transição entre um canal e outro, através de um sistema de \textit{fader}, no qual o volume de um canal é atenuado, enquanto o do outro obtém um ganho, até que a transição seja finalizada.
\par
Porém, esse tipo de sistema não leva em conta nuances da música ou qual seria o melhor momento para que a entrada da próxima música aconteça.
\par
Assim, o sistema proposto permite ao artista uma mixagem mais simples, sem grandes preocupações acerca do controle de bandas de frequência; modelo que pode facilitar mixagem para artistas que possuem uma biblioteca similar e consolidada ou jovens artistas que ainda não possuem total controle sobre as bandas de frequência mas que queiram experimentar.
\par
Para além de uma mixagem mais simples, o sistema propõe o uso de efeitos para que o artista tenha mais liberdade durante a criação de sua mixagem. Dessa forma, o sistema conta com a presença de dois efeitos: \textit{delay} e \textit{reverb}, que podem funcionar de forma automática; também controlada pelo botão central.
\par
A lógica de funcionamento do efeito leva em consideração a posição do botão central, já que quando o mesmo se encontra em uma posição na qual duas músicas estão sendo tocadas, entende-se que uma transição está acontecendo. Dessa forma, de forma automatizada, a presença do efeito parte do nulo até alcançar o seu máximo (na posição do meio do botão central) e conforme a posição do botão vai a uma extremidade, a presença do efeito diminui, pois se considera que a transição já foi realizada.
\par
Portanto, a proposta deste projeto visa implementar um sistema que leia dois sinais analógicos, correspondentes a sinais músicais advindo de toca-discos, CDJs ou qualquer equipamento que transmita sinais de áudio analógicos. Esses sinais serão processados em tempo real a partir de filtros passa-altas que são controlados por um botão central. Além disso, o sistema deve permitir ao usuário a possibilidade de selecionar qual efeito ele deseja utilizar, de forma que se selecione a quantidade da presença do efeito, possibilitando ao usuário rejeitar a presença do efeito quando a presença for nula.
\par
A implementação visa utilizar botões \textit{sliders} para o controle central e presença de efeitos e um botão \textit{on}/\textit{off} para a seleção do efeito. 

\section{Levantamento de Requisitos}

O levantamento de requisitos é uma etapa crucial no desenvolvimento de qualquer sistema, pois define as funcionalidades e características que ele deve possuir para atender às necessidades dos usuários finais. Nesta seção, são apresentados os requisitos funcionais e não funcionais, que especificam o comportamento esperado do sistema, assim como as restrições e qualidades que devem ser atendidas. Os requisitos foram organizados em categorias que abrangem desde aspectos de desempenho e interface do usuário até segurança e manutenção, assegurando uma visão completa e detalhada do que o sistema deve entrega

\subsection{Requisitos Funcionais}
\begin{itemize}
    \item O sistema deve permitir ao \textit{DJ} utilizar tanto CDJs quanto toca-discos.
    \item O sistema deve permitir ao \textit{DJ} controlar a presença dos dois canais.
    \item O sistema deve permitir ao \textit{DJ} controlar a presença dos efeitos.
\end{itemize}

\subsection{Requisitos Não Funcionais}
\begin{itemize}
    \item O sistema deve responder aos comandos do \textit{DJ} com latência mínima, garantindo uma experiência de mixagem fluida.
    \item A interface do usuário deve ser intuitiva e fácil de usar, permitindo que o \textit{DJ} faça ajustes rapidamente durante a performance.
    \item O sistema deve ser confiável e estável, capaz de lidar com longos períodos de uso contínuo sem falhas.
    \item O sistema deve oferecer uma qualidade de som de alta fidelidade, garantindo que o áudio reproduzido seja claro e com mínimas distorções.
\end{itemize}

\subsection{Requisitos de Interface do Usuário}
\begin{itemize}
    \item A interface do mixer deve incluir botões físicos ou controles táteis para ajuste de frequência e efeitos de áudio.
    \item A interface do mixer deve ser organizada de forma lógica e intuitiva, com controles agrupados por função para facilitar a navegação.
    \item A interface deve possuir indicações das funções dos botões de interação com o usuário.
\end{itemize}

\subsection{Requisitos de Sistema}
\begin{itemize}
    \item O sistema deve ser compatível com uma variedade de dispositivos de áudio externos, como CDJs e toca-discos.
    \item O sistema deve ser alimentado por uma fonte de energia padrão, como uma tomada elétrica.
    \item O sistema deve incluir interfaces de entrada e saída de áudio padrão RCA.
\end{itemize}

\subsection{Requisitos de Desempenho}
\begin{itemize}
    \item O sistema deve ser capaz de lidar com até dois canais de áudio simultaneamente, sem comprometer a qualidade do som ou a responsividade dos controles.
    \item O sistema deve suportar uma ampla gama de frequências de áudio, garantindo que os graves sejam reproduzidos com graves e os agudos sejam nítidos e claros.
\end{itemize}

\subsection{Requisitos de Segurança}
\begin{itemize}
    \item O sistema deve ser projetado para minimizar o risco de danos aos equipamentos de áudio conectados, oferecendo proteção contra sobrecarga ou curto-circuito.
\end{itemize}

\subsection{Requisitos de Manutenção}
\begin{itemize}
    \item O sistema deve ser projetado para facilitar a manutenção e reparo, com acesso fácil aos componentes internos e documentação clara sobre procedimentos de serviço.
\end{itemize}

\subsection{Requisitos de Compatibilidade}
\begin{itemize}
    \item O sistema deve ser capaz de se comunicar perfeitamente com qualquer dispositivo de reprodução de música profissional como canal de entrada e qualquer sistema de som de sáida. 
\end{itemize}

\section{Diagrama do Sistema}

\section{Fluxogramas de Sub-blocos}

\subsection{Bloco de Conversão AD}

\subsection{Bloco de Leitura do Botão Central}

\subsection{Bloco de Automação de Efeitos}

\subsection{Bloco de Efeitos - Delay}

\subsection{Bloco de Efeitos - Reverb}

\subsection{Bloco Mixer}

\subsection{Bloco de Conversão AD}

\section{Prova de Conceito}

Nessa seção, encontra-se uma implementação em um ambiente virtual no qual se pode simular a lógica de funcionamento do sistema.

	\subsection{\textit{PureData}}
    O \textit{PureData} \cite{puredata} é um ambiente de música computacional programável para análise, síntese e processamento de áudio através de sinais digitais em tempo real. 
    

    Esse ambiente permite, através dos seus blocos, a criação de sistemas de processamento de áudio com inúmeras funções implementadas, tanto pelos seus criadores quanto pela sua extensa comundiade. Nele, foi possível a criação de uma prova de conceito que engloba a lógica do botão central com o comando das frequências de corte, bem como o funcionamento dos efeitos. Para simular os sinais de entrada, utilizou-se arquivos WAV locais.

    Assim, a demonstração do sistema se divide em duas grandes funcionalidades: filtragem e efeitos.

	\subsection{Implementação de Filtragem}

    A filtragem lê dois arquivos de música no formato WAV, utilizando as funções \texttt{open}, \texttt{start} e \texttt{stop} para localização, execução e parada da reprodução, respectivamente. Em seguida, utilizou-se o comando \texttt{readsf\textasciitilde\ 2 1e+06} que configura a leitura dos sinais de forma estéreo e utilizando um milhão de amostras no seu \textit{buffer}. O mesmo processo é realizado para ambos arquivos. 

    Em seguida, utilizou-se a função \texttt{hip\textasciitilde}, que corresponde a aplicação de um filtro passa-altas. Porém, neste caso, o argumento que a função utiliza difere entre o canal 1 e 2. Conforme a Figura \ref{fig24}, o canal 1 ("FC do HPF1") recebe diretamente o parâmetro \textit{fc}, advindo do \textit{slider} em azul.
    
    Porém, o filtro passa-altas do canal 2 ("FC do HPF2") recebe um valor ajustado por uma expressão anterior, representada pela Equação \ref{eq:05}. Esse ajuste permite com que uma pequena variação na \textit{fc$_{1}$} promova uma grande variação em \textit{fc$_{2}$} e vice-versa. Além disso, em frequências centrais, a variação entre eles se torna mais similar. Essa expressão coincide com a descrição de um círculo de raio sendo a frequência de amostragem, que é o intervalo do botão central, centralizado no ponto (22050, 22050).

    \begin{equation}  \label{eq:05}
        fc_2 = 22050 - \sqrt{22050^2 - (fc - 22050)^2}
    \end{equation}

    Esse ajuste presente na Equação \ref{eq:05} tem a função de ponderar mudanças nas frequências pois mudanças lineares não são eficientes nesse caso de um controle centralizado, conforme se encontram as frequências de ambos canais na Figura \ref{fig45}.

    \begin{figure}[h]
        \centering
        \includegraphics[width=0.7\textwidth]{figuras/fig45.png}
        \caption{expressão para a fc$_{2}$}
        \label{fig45}
    \end{figure}

    \newpage
    Mudanças na casa de centenas no canal 1 causariam pouco efeito no canal 2 pois as baixas frequências possuem maior ganho em relação às altas frequências. Além disso, a mesma lógica pode ser aplicada no outro extremo do controle de frequência. 

    \begin{figure}[h]
        \centering
        \includegraphics[width=0.9\textwidth]{figuras/fig44.png}
        \caption{lógica de funcionamento do botão central no \textit{PureData}}
        \label{fig44}
    \end{figure}

    Na Figura \ref{fig44}, a frequência obtida do botão central, indicada como \( f_c \), varia de 0.2 a 22050 Hz. A frequência de corte do canal 1, referida como FC do HPF1, é equivalente a \( f_c \) do botão central. A frequência de corte do canal 2, identificada como FC do HPF2, é obtida pela Equação \ref{eq:05}. O sinal resultante, \texttt{send\textasciitilde\ music}, é a soma dos sinais filtrados dos canais 1 e 2.

	\subsection{Implementação de Efeitos}

    O funcionamento do efeito utiliza três parâmetros. Um botão \texttt{toogle}, cuja função é alternar entre os efeitos \textit{delay} e \textit{reverb}; um \textit{slider} que varia parâmetros internos dos efeitos e a frequência de corte do botão central, que automatiza o volume do efeito.

    O botão \texttt{toogle} visa alternar os efeitos. Possui duas posições, ou seja, sempre um efeito está ativo. Para alternar para o outro, basta alterar a posição. Figura \ref{fig46}.

    \begin{figure}[h]
        \centering
        \includegraphics[width=0.3\textwidth]{figuras/fig46.png}
        \caption{botão de seleção de efeito no \textit{PureData}}
        \label{fig46}
    \end{figure}

    O botão \textit{slider} visa alterar os parâmetros internos de cada efeito. Os valores variam de 0 a 1. O botão se encontra na Figura \ref{fig47}.

    \begin{figure}[h]
        \centering
        \includegraphics[width=0.2\textwidth]{figuras/fig47.png}
        \caption{botão de quantidade de efeito no \textit{PureData}}
        \label{fig47}
    \end{figure}

    Cada efeito utiliza o parâmetro \textit{fx} advindo do botão \textit{slider} e realiza uma adaptação para o seu parâmetro. No caso do \textit{reverb}, o valor de \textit{fx} é multiplicado por 100 e esse valro se torna a quantidade de dB que permanecerá na música após 1s. Para o \textit{delay}, esse valor é multiplicado por 1000 e se transforma no intervalo de tempo em ms da música que permanecerá no efeito.

    A lógica que permeia a seleção do efeito, encontra-se na Figura \ref{fig48}. Os comandos \texttt{receive\textasciitilde\ fx} são responsáveis por inserirem os efeitos como entrada. Cada volume do efeito é multiplicado pelo valor do \texttt{toogle}; um deles é multiplicado pelo valor atual enquanto o outro é multiplicado pelo inverso, de forma que o botão \texttt{toogle} funciona como um alternador para selecionar os efeitos. 

    \begin{figure}[h]
        \centering
        \includegraphics[width=0.3\textwidth]{figuras/fig48.png}
        \caption{botão de seleção de efeito no \textit{PureData}}
        \label{fig48}
    \end{figure}

    Nesse sistema, o volume do efeito é regido de forma automática em função da posição do botão central, ou seja, pelas frequências de corte. Na Figura \ref{fig49} é possível ver a variação do volume do efeito em função da frequência central.

    \begin{figure}[h]
        \centering
        \includegraphics[width=0.9\textwidth]{figuras/fig49.png}
        \caption{variação do volume dos efeitos no \textit{PureData}}
        \label{fig49}
    \end{figure}

    No \textit{PureData}, o bloco de automação de volume de efeitos é realizado utilizando as operações presentes na Figura \ref{fig50}.

    \begin{figure}[h]
        \centering
        \includegraphics[width=0.6\textwidth]{figuras/fig50.png}
        \caption{implementação da variação do volume dos efeitos no \textit{PureData}}
        \label{fig50}
    \end{figure}

    Ao final, o sinal dos efeitos é multiplicado pelo volume dos efeitos, e, posteriormente, somado ao sinal das filtragens, dando origem ao sinal de saída. Esse sinal é processado por um bloco de conversão de sinal digital em analógico e enfim foi reproduzido. Esse bloco de soma de sinais está representado na Figura \ref{fig51}.

    \begin{figure}[h]
        \centering
        \includegraphics[width=0.3\textwidth]{figuras/fig51.png}
        \caption{soma de sinais filtrado e de efeitos no \textit{PureData}}
        \label{fig51}
    \end{figure}

\section{Proposta de Implementação}

	\subsection{Implementação em \textit{Hardware}}

	\subsection{Implementação em \textit{Software}}

	\subsection{Integração}

\subsection{Diagramas de Subpartes}
\subsection{Diagrama de Integração}
\subsection{Diagrama de Comunicação}
\subsection{Fluxograma}
\subsection{Protótipo de Interface de Usuário}
\subsection{Documento de Especificação Técnica}
\subsection{Documento de Plano de Teste}