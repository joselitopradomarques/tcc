\chapter[Metodologia]{Metodologia}


\section{Proposta Geral}
A evolução dos equipamentos de mixagem se baseia na mixagem feita em vinil, adicionando novas funcionalidades ao longo do tempo, como o \textit{jogger} para atraso e avanço da música, \textit{fader} para volume e \textit{knobs} para frequências. Com o avanço da microeletrônica, esses equipamentos migraram para a eletrônica digital, utilizando \textit{DSPs} para processar formatos de música de alta qualidade como WAV e FLAC.

Atualmente, há dois tipos principais de equipamentos: caros, que não necessitam de computador; e mais baratos, que dependem dele mas oferecem melhor qualidade de processamento. DJs iniciantes geralmente têm dificuldades em realizar ajustes finos nas bandas de frequência em \textit{mixers} clássicos de três bandas. Esta habilidade é essencial para destacar elementos musicais e criar novas atmosferas. Quando um DJ já consolidou seu estilo, a mixagem torna-se mais mecânica, facilitando a transição entre músicas.

Este projeto propõe um \textit{mixer} com um controle central que manipula dois canais de entrada (duas músicas). Em vez dos tradicionais controles de três bandas de frequência, utilizam-se filtros passa-altas, onde o controle central ajusta simultaneamente as frequências de corte de ambos os canais, permitindo uma transição suave entre as músicas.

Além disso, o sistema incluirá dois efeitos, \textit{delay} e \textit{reverb}, controlados automaticamente pela posição do botão central. Quando o botão está na posição intermediária, os efeitos são aplicados ao máximo, diminuindo gradualmente conforme o botão se move para uma das extremidades.

Em suma, o sistema processará sinais de áudio analógicos em tempo real, utilizando filtros passa-altas e permitindo ao usuário selecionar e controlar a intensidade dos efeitos, com botões \textit{sliders} para os controles e um botão \textit{on}/\textit{off} para a seleção dos efeitos.

\section{Levantamento de Requisitos}

O levantamento de requisitos é uma etapa crucial no desenvolvimento de qualquer sistema, pois define as funcionalidades e características que ele deve possuir para atender às necessidades dos usuários finais. Nesta seção, são apresentados os requisitos funcionais e não funcionais, que especificam o comportamento esperado do sistema, assim como as restrições e qualidades que devem ser atendidas. Os requisitos foram organizados em categorias que abrangem desde aspectos de desempenho e interface do usuário até segurança e manutenção, assegurando uma visão completa e detalhada do que o sistema deve entregar.

\subsection{Requisitos Funcionais}
\begin{itemize}
    \item O sistema deve permitir ao \textit{DJ} utilizar tanto \textit{CDJs} quanto toca-discos.
    \item O sistema deve permitir ao \textit{DJ} controlar a presença dos dois \textit{canais}.
    \item O sistema deve permitir ao \textit{DJ} controlar a presença dos efeitos.
\end{itemize}

\subsection{Requisitos Não-Funcionais}
\begin{itemize}
    \item O sistema deve responder aos comandos do \textit{DJ} com latência mínima, garantindo uma experiência de mixagem fluida.
    \item A interface do usuário deve ser intuitiva e fácil de usar, permitindo que o \textit{DJ} faça ajustes rapidamente durante a performance.
    \item O sistema deve ser confiável e estável, capaz de lidar com longos períodos de uso contínuo sem falhas.
    \item O sistema deve oferecer uma qualidade de som de alta fidelidade, garantindo que o áudio reproduzido seja claro e com mínimas distorções.
    \item A interface do mixer deve incluir botões físicos ou controles táteis para ajuste de frequência e efeitos de áudio.
    \item A interface do mixer deve ser organizada de forma lógica e intuitiva, com controles agrupados por função para facilitar a navegação.
    \item A interface deve possuir indicações das funções dos botões de interação com o usuário.
    \item O sistema deve ser compatível com uma variedade de dispositivos de áudio externos, como \textit{CDJs} e toca-discos.
    \item O sistema deve ser alimentado por uma fonte de energia padrão, como uma tomada elétrica.
    \item O sistema deve incluir interfaces de entrada e saída de áudio padrão \textit{RCA}.
    \item O sistema deve ser capaz de lidar com até dois \textit{canais} de áudio simultaneamente, sem comprometer a qualidade do som ou a responsividade dos controles.
    \item O sistema deve suportar uma ampla gama de frequências de áudio, garantindo que os graves sejam reproduzidos com profundidade e os agudos sejam nítidos e claros.
    \item O sistema deve ser projetado para minimizar o risco de danos aos equipamentos de áudio conectados, oferecendo proteção contra sobrecarga ou curto-circuito.
    \item O sistema deve ser projetado para facilitar a manutenção e reparo, com acesso fácil aos componentes internos e documentação clara sobre procedimentos de serviço.
    \item O sistema deve ser capaz de se comunicar perfeitamente com qualquer dispositivo de reprodução de música profissional como canal de entrada e qualquer sistema de som de saída. 
\end{itemize}


\section{Fluxograma do Mixer}

O \textit{mixer} proposto possui subblocos de funcionamento e se interligam conforme se encontra na Figura \ref{fig52}.

De forma geral, o sistema funciona em um laço contínuo de leitura de sinais, tanto das músicas quanto dos controles, e modificações de parâmetros para que os sinais sejam processados, e, enfim, reproduzidos. Assim, cada ciclo pode ser representado na Fig. \ref{fig52}.

\begin{figure}[h]
    \centering
    \includegraphics[width=\textwidth]{figuras/fig52.png}
    \caption{fluxograma geral do \textit{mixer}}
    \label{fig52}
\end{figure}

Assim, a partir dos dispositivos de reprodução de música, os sinais adentrarão o sistema, de forma que serão convertidos para sinais digitais. Enquanto isso, é feita uma varredura dos valores correntes dos controles disponíveis na interface.

Conforme se adquire valores para os parâmetros, tanto a filtragem de sinais quanto o processamento dos efeitos são realizados novamente a fim de ajustarem seu comportamento conforme os novos valores dos controles.

Por fim, os sinais da filtragem e do efeito são combinados e convertidos para analógico, de forma que possam ser reproduzidos ao final.

Nas subseções abaixo, cada bloco presente no Fluxograma Geral do Sistema, Fig. \ref{fig52}, será detalhado sobre o seu funcionamento conceitual proposto através de um subfluxograma, que detalha o processamento interno.

\subsection{Bloco de Leitura de Sinais}

Os sinais analógicos que são lidos incluem os sinais de música; e os sinais de controle que são advindos de dois potenciômetros e de um botão de duas posições, respectivamente para obtenção da frequência central, da quantidade de efeito desejado presente no processamento e o de escolha de qual efeito se deseja utilizar. 

\begin{figure}[h]
    \centering
    \includegraphics[width=\textwidth]{figuras/fig54.png}
    \caption{bloco de leitura de sinais analógicos}
    \label{fig54}
\end{figure}

As operações realizadas nesses sinais são visualizadas na Figura \ref{fig54}.

\subsection{Bloco de Filtragem}

No bloco de filtragem, os sinais já estão no domínio digital. Assim, deve-se ler a posição do botão central. Esse sinal é adquirido através da conversão de um sinal analógico advindo de um potenciômetro e convertido, de um sinal elétrico para um digital. 

Assim, pode-se obter a posição na qual o botão se encontra. Essa posição, que se encontrará em um intervalo de valores quantizados, será normalizado e convertido para um valor de frequência de corte, ou seja, estará entre 20 e 22050 Hz.

Com esse valor de frequência de corte, o filtro passa-altas do canal 1 será atualizado; e seu sinal filtrado. 

No canal 2, um novo valor para a frequência de corte 2 será obtido através da expressão de Equação \ref{eq:05}. Assim, o filtro passa-altas deste canal será atualizado conforme a nova frequência de corte, como se encontra na Figura \ref{fig55}. 

No bloco de filtragem, todos os sinais analógicos já estão codificados de forma que podem ser processados digitalmente. Assim, tem-se a \textit{fc} como o valor de frequência entre 20 a 22050 Hz. Dessa forma, valores são atribuídos aos parâmetros \textit{fc$_{1}$} e \textit{fc$_{2}$}.

\begin{figure}[h]
    \centering
    \includegraphics[width=\textwidth]{figuras/fig55.png}
    \caption{bloco de filtragem}
    \label{fig55}
\end{figure}

No subfluxograma da Figura \ref{fig55}, \textit{fc} corresponde à frequência de corte central, ou seja, àquela lida e convertida do botão central; \textit{fc$_{1}$} é a frequência de corte para o filtro passa-altas 1 (HPF1) e \textit{fc$_{2}$} é a frequência de corte do canal 2 para o filtro passa-altas 2 (HPF2).

Ao final, os dois canais são combinados e se tem o sinal \textit{signal\_master}, que é mandado tanto para o bloco de processamento de efeitos quanto o bloco final de combinação dos dois sinais: de efeito e de combinação dos canais 1 e 2.

\subsection{Bloco de Efeitos}

Os efeitos do \textit{mixer} podem ter seus parâmetros de quantidade de reverberação após 1s e o intervalo de atraso (em milissegundos) de amostras configurados conforme o botão de quantidade de efeito. 

\begin{figure}[h]
    \centering
    \includegraphics[width=\textwidth]{figuras/fig56.png}
    \caption{bloco de efeitos}
    \label{fig56}
\end{figure}

Além disso, o usuário poderá escolher qual efeito deseja utilizar através de um botão de duas posições, conforme a Figura \ref{fig56}. Os dois parâmetros de seleção e quantidade de efeitos são obtidos do bloco de leitura de sinais. Ao fim desse bloco, tem-se o sinal isolado do efeito, ou seja, o sinal obtido ao final do bloco \textit{mixer} com o efeito aplicado, atribuído ao \textit{signal\_fx}. 

\subsection{Bloco de Automação de Efeitos}

Nesse \textit{mixer} proposto, o volume do efeito é ajustado automaticamente, conforme a frequência central presente na Figura \ref{fig57}. Assim, quando a \textit{fc} está nas extremidades, o volume do efeito é nulo. Porém, ele obtém um ganho conforme a \textit{fc} alcança a posição central da banda de frequência.

\begin{figure}[h]
    \centering
    \includegraphics[width=\textwidth]{figuras/fig57.png}
    \caption{bloco de automação do volume de efeitos}
    \label{fig57}
\end{figure}

Para a obtenção do ganho do efeito, uma expressão é utilizada para a conversão do valor da \textit{fc} em \textit{volume\_fx}, que é o ganho do \textit{signal\_fx}.

\subsection{Bloco Mixer}

A operação de somar dois sinais recebe o nome de \textit{mixing}. Dessa forma, entende-se que nesse processo há dois momentos em que os sinais são misturados. O primeiro é quando os sinais dos canais 1 e 2 são misturados. Outro momento acontece quando os sinais já misturados dos canais 1 e 2 são misturados com o sinal do efeito. Nesse caso, esse bloco se refere à mistura final, ou seja, do segundo caso.

\begin{figure}[h]
    \centering
    \includegraphics[width=\textwidth]{figuras/fig58.png}
    \caption{bloco de mistura final de sinais}
    \label{fig58}
\end{figure}

No subfluxograma da Figura \ref{fig58}, o sinal do efeito é atenuado ou amplificado conforme o volume obtido pelo bloco de automação de volume de efeitos. Assim, esse sinal de efeito com volume ajustado é misturado com o sinal obtido do bloco de filtragem, de forma que o sinal de saída, \textit{signal\_output}, é obtido. Porém, esse sinal ainda se encontra no domínio digital.

\subsection{Bloco de Conversão AD}

O sinal de saída obtido pelo bloco \textit{mixer} precisa ser convertido para um sinal analógico para que possa ser reproduzido em um sistema de som.

\begin{figure}[h]
    \centering
    \includegraphics[width=\textwidth]{figuras/fig59.png}
    \caption{bloco de conversão digital analógico}
    \label{fig59}
\end{figure}

Dessa forma, conforme a Figura \ref{fig59}, o sinal digital passará por processos de conversão digital analógico, filtragem e amplificação da sua potência para que enfim possa ser reproduzido por caixas de som.


\newpage
\section{Prova de Conceito}

Nessa seção, encontra-se uma implementação em um ambiente virtual no qual se pode simular a lógica de funcionamento do sistema.

\subsection{\textit{PureData}}
O \textit{PureData} \cite{puredata} é um ambiente de música computacional programável para análise, síntese e processamento de áudio através de sinais digitais em tempo real.

Esse ambiente permite, através dos seus blocos, a criação de sistemas de processamento de áudio com inúmeras funções implementadas, tanto pelos seus criadores quanto pela sua extensa \textit{comundiade}. Nele, foi possível a criação de uma prova de conceito que engloba a lógica do botão central com o comando das frequências de corte, bem como o funcionamento dos efeitos. Para simular os sinais de entrada, utilizou-se arquivos WAV locais.

Assim, a demonstração do sistema se divide em duas grandes funcionalidades: filtragem e efeitos.

\subsection{Implementação de Filtragem}

A filtragem lê dois arquivos de música no formato WAV, utilizando as funções \texttt{open}, \texttt{start} e \texttt{stop} para localização, execução e parada da reprodução, respectivamente. Em seguida, utilizou-se o comando \texttt{readsf\textasciitilde\ 2 1e+06} que configura a leitura dos sinais de forma estéreo e utilizando um milhão de amostras no seu \textit{buffer}. O mesmo processo é realizado para ambos arquivos.

Em seguida, utilizou-se a função \texttt{hip\textasciitilde}, que corresponde a aplicação de um filtro passa-altas. Porém, neste caso, o argumento que a função utiliza difere entre o canal 1 e 2. Conforme a Figura \ref{fig24}, o canal 1 ("FC do HPF1") recebe diretamente o parâmetro \textit{fc}, advindo do \textit{slider} em azul.

Porém, o filtro passa-altas do canal 2 ("FC do HPF2") recebe um valor ajustado por uma expressão anterior, representada pela Equação \ref{eq:05}. Esse ajuste permite com que uma pequena variação na \textit{fc$_{1}$} promova uma grande variação em \textit{fc$_{2}$} e vice-versa. Além disso, em frequências centrais, a variação entre eles se torna mais similar. Essa expressão coincide com a descrição de um círculo de raio sendo a frequência de amostragem, que é o intervalo do botão central, centralizado no ponto (22050, 22050).

\begin{equation}  \label{eq:05}
    fc_2 = 22050 - \sqrt{22050^2 - (fc - 22050)^2}
\end{equation}

Esse ajuste presente na Equação \ref{eq:05} tem a função de ponderar mudanças nas frequências pois mudanças lineares não são eficientes nesse caso de um controle centralizado, conforme se encontram as frequências de ambos canais na Figura \ref{fig45}.

\begin{figure}[h]
    \centering
    \includegraphics[width=0.7\textwidth]{figuras/fig45.png}
    \caption{expressão para a \textit{fc$_{2}$}}
    \label{fig45}
\end{figure}

Mudanças na casa de centenas no canal 1 causariam pouco efeito no canal 2 pois as baixas frequências possuem maior ganho em relação às altas frequências. Além disso, a mesma lógica pode ser aplicada no outro extremo do controle de frequência.

\begin{figure}[h]
    \centering
    \includegraphics[width=0.9\textwidth]{figuras/fig44.png}
    \caption{lógica de funcionamento do botão central no \textit{PureData}}
    \label{fig44}
\end{figure}

Na Figura \ref{fig44}, a frequência obtida do botão central, indicada como \( f_c \), varia de 0.2 a 22050 Hz. A frequência de corte do canal 1, referida como FC do HPF1, é equivalente a \( f_c \) do botão central. A frequência de corte do canal 2, identificada como FC do HPF2, é obtida pela Equação \ref{eq:05}. O sinal resultante, \texttt{send\textasciitilde\ music}, é a soma dos sinais filtrados dos canais 1 e 2.

\subsection{Implementação de Efeitos}

O funcionamento do efeito utiliza três parâmetros. Um botão \texttt{toggle}, cuja função é alternar entre os efeitos \textit{delay} e \textit{reverb}; um \textit{slider} que varia parâmetros internos dos efeitos e a frequência de corte do botão central, que automatiza o volume do efeito.

O botão \texttt{toggle} visa alternar os efeitos. Possui duas posições, ou seja, sempre um efeito está ativo. Para alternar para o outro, basta alterar a posição. Figura \ref{fig46}.

\begin{figure}[h]
    \centering
    \includegraphics[width=0.3\textwidth]{figuras/fig46.png}
    \caption{botão de seleção de efeito no \textit{PureData}}
    \label{fig46}
\end{figure}

O botão \textit{slider} visa alterar os parâmetros internos de cada efeito. Os valores variam de 0 a 1. O botão se encontra na Figura \ref{fig47}.

\begin{figure}[h]
    \centering
    \includegraphics[width=0.2\textwidth]{figuras/fig47.png}
    \caption{botão de quantidade de efeito no \textit{PureData}}
    \label{fig47}
\end{figure}

Cada efeito utiliza o parâmetro \textit{fx} advindo do botão \textit{slider} e realiza uma adaptação para o seu parâmetro. No caso do \textit{reverb}, o valor de \textit{fx} é multiplicado por 100 e esse valor se torna a quantidade de \textit{dB} que permanecerá na música após 1s. Para o \textit{delay}, esse valor é multiplicado por 1000 e se transforma no intervalo de tempo em \textit{ms} da música que permanecerá no efeito.

A lógica que permeia a seleção do efeito, encontra-se na Figura \ref{fig48}. Os comandos \texttt{receive\textasciitilde\ fx} são responsáveis por inserirem os efeitos como entrada. Cada volume do efeito é multiplicado pelo valor do \texttt{toggle}; um deles é multiplicado pelo valor atual enquanto o outro é multiplicado pelo inverso, de forma que o botão \texttt{toggle} funciona como um alternador para selecionar os efeitos.

\begin{figure}[h]
    \centering
    \includegraphics[width=0.3\textwidth]{figuras/fig48.png}
    \caption{botão de seleção de efeito no \textit{PureData}}
    \label{fig48}
\end{figure}

Nesse sistema, o volume do efeito é regido de forma automática em função da posição do botão central, ou seja, pelas frequências de corte. Na Figura \ref{fig49} é possível ver a variação do volume do efeito em função da frequência central.

\begin{figure}[h]
    \centering
    \includegraphics[width=0.9\textwidth]{figuras/fig49.png}
    \caption{variação do volume dos efeitos no \textit{PureData}}
    \label{fig49}
\end{figure}

No \textit{PureData}, o bloco de automação de volume de efeitos é realizado utilizando as operações presentes na Figura \ref{fig50}.

\begin{figure}[h]
    \centering
    \includegraphics[width=0.6\textwidth]{figuras/fig50.png}
    \caption{implementação da variação do volume dos efeitos no \textit{PureData}}
    \label{fig50}
\end{figure}

Ao final, o sinal dos efeitos é multiplicado pelo volume dos efeitos, e, posteriormente, somado ao sinal das filtragens, dando origem ao sinal de saída. Esse sinal é processado por um bloco de conversão de sinal digital em analógico e enfim foi reproduzido. Esse bloco de soma de sinais está representado na Figura \ref{fig51}.

\begin{figure}[h]
    \centering
    \includegraphics[width=0.3\textwidth]{figuras/fig51.png}
    \caption{soma de sinais filtrado e de efeitos no \textit{PureData}}
    \label{fig51}
\end{figure}


\section{Proposta de Implementação}

\subsection{Implementação em Hardware}

Texto geral sobre a implementação em hardware percorrendo todos pontos abaixo

\paragraph{Conectores}
Para a aquisição dos sinais de áudio, o \textit{mixer} contará com conectores \textit{RCA} devido à padronização imposta pela indústria e pela facilidade de se ter cabos e entradas desse tipo, Figura \ref{fig22}.

Dessa forma, dois \textit{RCA}s serão utilizados para cada canal, sendo assim: 4 \textit{RCA}s para entrada e 2 \textit{RCA}s para a saída, totalizando 6 \textit{RCA}s.

\paragraph{Botões}

A interação com o usuário é necessária para a obtenção de parâmetros como frequência de corte, quantidade de efeitos e a seleção do efeito desejado. Para isso, dois \textit{sliders} serão utilizados e uma chave de duas posições, respectivamente. 

Os \textit{sliders} em si possuem um potenciômetro. Dessa forma, são alimentados por uma tensão e sua posição é obtida através de uma tensão de saída, que é proporcional à tensão de entrada. A preferência por esse tipo de botão se dá à precisão alcançada utilizando dois dedos, o que confere à mudança de frequência precisão. Dessa forma, o botão da Figura \ref{fig61} é esperado para que o usuário interaja com o \textit{mixer}. 

\begin{figure}[h]
    \centering
    \includegraphics[width=0.5\textwidth]{figuras/fig61.png}
    \caption{botão \textit{slider} horizontal para frequência central \cite{robocore}}
    \label{fig61}
\end{figure}

Em contrapartida, a chave de duas posições gera dois níveis possíveis de tensões, tensão nula ou de alimentação. Cada nível deve se relacionar a um efeito desejado. Dessa forma, um botão como o da Figura \ref{fig62} é esperado para a interface com o usuário.

Para a escolha de quantidade de efeitos desejado, um botão semelhante ao acima será escolhido para a interface do usuário.

\begin{figure}[h]
    \centering
    \includegraphics[width=0.3\textwidth]{figuras/fig62.png}
    \caption{botão de duas posições para seleção do efeito \cite{evea}}
    \label{fig62}
\end{figure}

\paragraph{Conversão AD}

A conversão analógica digital será utilizada em duas categorias de sinais: de áudio e de controle. Então, dois tipos de níveis de quantização são esperados: um alto para que as músicas convertidas tenham definição e outro baixo devido à precisão necessária para que se infira qual efeito é desejado (dois níveis) ou em qual frequência central se deseja estar.

Dessa forma, para a conversão dos sinais analógicos responsáveis pela aquisição de sinais de áudio, utilizar-se-á conversores de 16 \textit{bits}, semelhante ao da Figura \ref{fig63}.

\begin{figure}[h]
    \centering
    \includegraphics[width=0.5\textwidth]{figuras/fig63.jpg}
    \caption{conversor analógico digital \cite{walmartRobotHuman}}
    \label{fig63}
\end{figure}    

Para a obtenção dos sinais de controle, utilizar-se-á a conversão analógico digital advinda de um \textit{Arduino Uno}, que possui uma resolução de 10 \textit{bits}. O microcontrolador que será utilizado se encontra na Figura \ref{fig64}.

\begin{figure}[h]
    \centering
    \includegraphics[width=0.5\textwidth]{figuras/fig64.jpeg}
    \caption{\textit{Arduino Uno} para conversão de sinais de controle \cite{vidadesilicioPlacaCabo}}
    \label{fig64}
\end{figure}    

\paragraph{Protocolos de Comunicação}
Ao se utilizar adequados conversores analógico digitais, os sinais podem ser adquiridos e interpretados pela \textit{Raspberry Pi}. Porém, ainda há inúmeras formas de se adquirir esses sinais analógicos a partir de um dispositivo. Uma forma eficiente é a utilização de protocolos de comunicação serial, que possibilitam a transmissão e/ou recepção de sinais através de um canal de comunicação.

Dessa forma, para essa aplicação, pode-se utilizar um canal de transmissão para a recepção de sinais de áudio e outro para os sinais de controle. Para isso, deve-se equiparar os protocolos para a transmissão e recepção.

Assim, para a recepção de sinais de áudio, o \textit{mixer} utilizará I2C pois os conversores levantados utilizam esse protocolo. Além disso, para a recepção dos sinais de controle, os protocolos disponíveis podem ser UART (\textit{Asynchronous Receiver/Transmitter}), I2C (\textit{Inter-Integrated Circuit}) ou SPI (\textit{Serial Peripheral Interface}).

\paragraph{Unidades de Processamento}

A unidade de processamento é o local ou dispositivo no qual os processamentos de sinais serão realizados. Para isso, utilizar-se-á uma \textit{Raspberry Pi} devido ao seu poder de processamento e às suas variadas possibilidades de compilação quanto à linguagem de processamento. Esse dispositivo pode ser visualizado na Figura \ref{fig65}.

\begin{figure}[h]
    \centering
    \includegraphics[width=0.5\textwidth]{figuras/fig65.jpg}
    \caption{\textit{Raspberry Pi} para processamento dos sinais \cite{adrenalineDisplaysLanar}}
    \label{fig65}
\end{figure}    

\paragraph{Conversão DA}

A conversão digital analógico é necessária para que o sinal final, após o processamento, possa ser reproduzido por sistemas de áudio que utilizam sinais analógicos. Porém, a \textit{Raspberry Pi} conta, nativamente, com uma saída/entrada de áudio com conector 3.5mm, conforme a Figura \ref{fig20}.

\subsection{Implementação em Software}

Na \textit{Raspberry Pi}, com os sinais convertidos e adquiridos, é necessário que haja o processamento para que, ao final, haja o sinal processado. Assim, devido à velocidade de processamento, a linguagem C será utilizada para a implementação do sistema. 
    
\subsection{Protótipo de Interface de Usuário}

Assim, de forma similar a um \textit{mixer} atual, a interface do usuário seria implementada com dois botões \textit{sliders} horizontais com um botão de duas posições. Na parte traseira, haverá os conectores \textit{RCA} e de alimentação.
