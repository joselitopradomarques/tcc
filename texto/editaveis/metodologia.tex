\chapter[Metodologia]{Metodologia}


\section{Proposta Geral}
Ao analisar a história de mixers e equipamentos e o mercado atual, chega-se a conclusão de que a evolução dos equipamentos foi e é realizada partindo de um modelo que se baseia na mixagem feita em vinil. Ao longo do tempo, acrescentou-se novas funcionalidades e formas de interações, porém, na maioria das vezes, mantendo o jogger para atraso e avanço da música, \textit{fadder} para volume e knobs para frequências.
\par
Além disso, conforme a evolução da microeletrônica, equipamentos começaram a migrar aos poucos para a eletrônica digital, culminando na utilização de um DSP para o processamento do sinal de áudio para comportar formatos de música com baixa compressão como WAV e FLAC. 
\par
Assim, o que se vê hoje são dois extremos: equipamentos extremamente caros mas que não necessitam da presença de um computador e equipamentos dependentes, que são mais baratos e acabam possuindo melhor qualidade nos processamentos, que dificultam a mobilidade.
\par
Além disso, em experiência adquiridas com \textit{DJs}, observa-se que uma das maiores dificuldades que iniciantes possuem é realizar um ajuste fino entre as bandas de frequências quando lidam com \textit{mixers} clássicos de ajustes de três bandas. E é essa habilidade que permite o artista de destacar elementos de músicas e criar uma nova atmosfera. Apesar disso, quando o estilo de gênero de um \textit{DJ} está consolidad em uma área, a mixagem se torna mais mecânica, de forma que os elementos entre as música selecionadas se tornam semelhantes, o que permite maior flexibilidade na alteração de parâmetros de mixagem. Ou seja, há menos margem para que uma colagem sonora feita se torne desagradável aos ouvidos.
\par
Dessa forma, o intuito é conceber um \textit{mixer} que conte com um controle central, que visa controlar duas músicas, ou dois \textit{decks}. Em contrapartida à estrutura clássica de mixagem, que conta com controles de ganho de três bandas de frequência para cada canal, a proposta visa a utilização de um filtro passa-altas para cada canal. 
\par
O dispostivo conta com dois canais de entrada, ou seja, duas músicas de entrada. Conforme um filtro está passando toda a banda de um canal, o outro filtro automaticamente está rejeitando o outro canal. E conforme o controle central avança em direção ao canal dois, a frequência de corte do primeiro canal aumenta, diminuindo a presença da música um, enquanto a frequência de corte do canal dois diminui, aumentando a presença de elementos da música dois. 
\par
Assim, um controle central de frequência comanda ambas frequências de corte. Quando esse controle está em um extremo, apenas um canal é transmitido para a saída. Já no outro extremo, apenas o outro canal é transmitido. E conforme o controle avança de um extremo a outro, uma combinação entre as duas músicas é transmitida à saída.
\par
Alguns equipamentos atuais contam com ferramentas de uma mixagem automática, na qual uma lista de músicas é selecionada. O próprio sistema realiza uma sincronia de tons e realiza a transição entre um canal e outro, através de um sistema de \textit{fader}, no qual o volume de um canal cai, enquanto do outro aumenta, até que a transição para a segunda música esteja finalizada.
\par
Porém, esse tipo de sistema não leva em conta nuances da música ou qual seria o melhor momento para que a entrada da próxima música aconteça.
\par
Assim, o sistema proposto permite ao artista uma mixagem mais simples, sem grandes preocupações acerca do controle de bandas de frequência; modelo que pode facilitar mixagem para artistas que possuem uma biblioteca similar e consolidada ou jovens artistas que ainda não possuem total controle sobre as bandas de frequência mas que queiram experimentar.
\par
Para além de uma mixagem mais simples, o sistema propõe o uso de efeitos para que o artista tenha mais liberdade durante a criação de sua mixagem. Dessa forma, o sistema conta com a presença de dois efeitos: \textit{delay} e \textit{reverb}, que podem funcionar de forma automática; também controlada pelo botão central.
\par
A lógica de funcionamento do efeito leva em consideração a posição do botão central, já que quando o mesmo se encontra em uma posição na qual duas músicas estão sendo tocadas, entende-se que uma transição está acontecendo. Dessa forma, de forma automatizada, a presença do efeito parte do nulo até alcançar o seu máximo (na posição do meio do botão central) e conforme a posição do botão vai a uma extremidade, a presença do efeito diminui, pois se considera que a transição já foi realizada.
\par
Portanto, a proposta deste projeto visa implementar um sistema que leia dois sinais analógicos, correspondentes a sinais músicais advindo de toca-discos, CDJs ou qualquer equipamento que transmita sinais de áudio analógicos. Esses sinais serão processados em tempo real a partir de filtros passa-altas que são controlados por um botão central. Além disso, o sistema deve permitir ao usuário a possibilidade de selecionar qual efeito ele deseja utilizar, de forma que se selecione a quantidade da presença do efeito, possibilitando ao usuário rejeitar a presença do efeito quando a presença for nula.
\par
A implementação visa utilizar botões \textit{sliders} para o controle central e presença de efeitos e um botão \textit{on}/\textit{off} para a seleção do efeito. 

\section{Levantamento de Requisitos}

O levantamento de requisitos é uma etapa crucial no desenvolvimento de qualquer sistema, pois define as funcionalidades e características que ele deve possuir para atender às necessidades dos usuários finais. Nesta seção, são apresentados os requisitos funcionais e não funcionais, que especificam o comportamento esperado do sistema, assim como as restrições e qualidades que devem ser atendidas. Os requisitos foram organizados em categorias que abrangem desde aspectos de desempenho e interface do usuário até segurança e manutenção, assegurando uma visão completa e detalhada do que o sistema deve entrega

\subsection{Requisitos Funcionais}
\begin{itemize}
    \item O sistema deve permitir ao \textit{DJ} utilizar tanto CDJs quanto toca-discos.
    \item O sistema deve permitir ao \textit{DJ} controlar a presença dos dois canais.
\end{itemize}

\subsection{Requisitos Não Funcionais}
\begin{itemize}
    \item O sistema deve responder aos comandos do \textit{DJ} com latência mínima, garantindo uma experiência de mixagem fluida.
    \item A interface do usuário deve ser intuitiva e fácil de usar, permitindo que o \textit{DJ} faça ajustes rapidamente durante a performance.
    \item O sistema deve ser confiável e estável, capaz de lidar com longos períodos de uso contínuo sem falhas.
    \item O sistema deve oferecer uma qualidade de som de alta fidelidade, garantindo que o áudio reproduzido seja claro e com mínimas distorções.
\end{itemize}

\subsection{Requisitos de Interface do Usuário}
\begin{itemize}
    \item A interface do mixer deve incluir botões físicos ou controles táteis para ajuste de frequência e efeitos de áudio.
    \item A interface do mixer deve ser organizada de forma lógica e intuitiva, com controles agrupados por função para facilitar a navegação.
    \item O sistema deve permitir ao \textit{DJ} personalizar a aparência e layout da interface de usuário conforme suas preferências.
    \item O sistema deve ser compatível com dispositivos de entrada externos, como controladores MIDI ou dispositivos de toque, para suportar diferentes estilos de interação.
\end{itemize}

\subsection{Requisitos de Sistema}
\begin{itemize}
    \item O sistema deve ser compatível com uma variedade de dispositivos de áudio externos, como CDJs e toca-discos.
    \item O sistema deve ser alimentado por uma fonte de energia padrão, como uma tomada elétrica.
    \item O sistema deve incluir interfaces de entrada e saída de áudio padrão, como conectores RCA ou XLR, para facilitar a conexão com outros equipamentos de áudio.
\end{itemize}

\subsection{Requisitos de Desempenho}
\begin{itemize}
    \item O sistema deve ser capaz de lidar com até dois canais de áudio simultaneamente, sem comprometer a qualidade do som ou a responsividade dos controles.
    \item O sistema deve suportar uma ampla gama de frequências de áudio, garantindo que os graves sejam reproduzidos com punch e os agudos sejam nítidos e claros.
\end{itemize}

\subsection{Requisitos de Segurança}
\begin{itemize}
    \item O sistema deve ser projetado para minimizar o risco de danos aos equipamentos de áudio conectados, oferecendo proteção contra sobrecarga ou curto-circuito.
\end{itemize}

\subsection{Requisitos de Manutenção}
\begin{itemize}
    \item O sistema deve ser projetado para facilitar a manutenção e reparo, com acesso fácil aos componentes internos e documentação clara sobre procedimentos de serviço.
\end{itemize}

\subsection{Requisitos de Compatibilidade}
\begin{itemize}
    \item O sistema deve ser compatível com uma variedade de formatos de áudio comuns, como WAV, MP3 e FLAC.
\end{itemize}

\subsection{Diagrama de Subpartes}

\subsection{Fluxograma de Sub-blocos}

\section{Prova de Conceito}

	\subsection{\textit{PureData}}

	\subsection{Implementação de Mixagem}

	\subsection{Implementação de Efeitos}

\section{Proposta de Implementação}

	\subsection{Implementação em \textit{Hardware}}

	\subsection{Implementação em \textit{Software}}

	\subsection{Integração}

\subsection{Diagramas de Subpartes}
\subsection{Diagrama de Integração}
\subsection{Diagrama de Comunicação}
\subsection{Fluxograma}
\subsection{Protótipo de Interface de Usuário}
\subsection{Documento de Especificação Técnica}
\subsection{Documento de Plano de Teste}